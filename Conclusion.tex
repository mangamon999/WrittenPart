\section{Conclusão}

Foi proposto e desenvolvido neste trabalho um protocolo para a camada de controle de acesso ao meio em redes de sensores sem fio denominado CS-MAC, assim chamado por se tratar de uma extensão do protocolo existente S-MAC. 

O protocolo CS-MAC foi desenvolvido com o objetivo de promover uma redução na latência em transmissões entre nós distantes na rede de sensores em relação ao protocolo original S-MAC e se mostrou eficiente conforme se pode comprovar a partir da análise dos resultados obtidos através dos experimentos computacionais realizados.

O trabalho demonstrou que o protocolo proposto conseguiu alcançar uma redução significativa na latência em relação ao protocolo S-MAC, satisfazendo as expectativas iniciais.

O CS-MAC apresentou uma redução média da latência em torno de  $25\%$ em relação ao S-MAC, chegando a mais de $58\%$ nos casos de melhor desempenho. O protocolo produziu um efeito negativo ao aumentar o número de transmissões realizadas em $17,3\%$ em média, porém demonstrou-se que esse efeito poderia ser amenizado através de alterações nas configurações do protocolo.

Para trabalhos futuros podem ser considerados:

\begin{itemize}

\item \textbf{Estruturas diferenciadas de \emph{backbone}:} construção de vias rápidas de transmissão com estruturas diferentes para avaliação de desempenho e capacidade de atender a demandas diferenciadas em relação ao fluxo de pacotes.

\item \textbf{Uso de agendas diferenciadas:} avaliação do uso de agendas com diferentes períodos de atividade e inatividade com o objetivo de avaliar o impacto no consumo de energia.

\item \textbf{Uso de protocolos diferenciados:} utilização de protocolos diferentes para a sincronização dos nós fora do backbone e avaliação de seus impactos.

\end{itemize}