\section{Controle de acesso ao meio}
\label{sec:macProtocols}

Protocolos de controle de acesso ao meio (MAC) são uma subcamada da DLL (\emph{data link layer}) que é a primeira camada de protocolo acima da camada física. Eles são responsáveis por prover endereçamento e acesso ao canal de comunicação. Segundo \citeauthoronline{Karl2005} (\citeyear{Karl2005}), a tarefa principal de qualquer protocolo MAC é regular o acesso de um número de nós a um meio de comunicação compartilhado, de tal modo que os requisitos de desempenho de cada aplicação sejam satisfeitos.

Segundo \citeauthoronline{Ye2002} (\citeyear{Ye2002}), para desenvolver protocolos para redes de sensores, é preciso considerar como atributos principais a serem alcançados: eficiência no consumo de energia para aumentar a vida útil da rede, dado às dificuldades de substituição das baterias; escalabilidade para adaptação a mudanças no tamanho, densidade e topologia da rede, uma vez que alguns nós da rede irão ``morrer'' com o passar do tempo, novos nós podem ser inseridos na rede mais tarde ou ainda alguns nós podem se mover para locais diferentes. Outros atributos como justiça na distribuição de recursos (por exemplo o meio de comunicação), latência, e ritmo de transferência de dados que são tidos como principais em outros tipos de redes sem fio são aqui tidos como secundários.

%---------------------------------------------------------------------
\subsection{Padrões de comunicação em RSSFs}
\label{sec:comnPatt}
%-------------------------------------------
 
Segundo \citeauthoronline{Demirkol2006} (\citeyear{Demirkol2006}), os tipos de padrões de comunicação em RSSFs determinam o comportamento do tráfego de dados na redes de sensores que deve ser tratado pelos protocolos da camada MAC. 

\citeauthoronline{Kulkarni2004} (\citeyear{Kulkarni2004}) define três tipos de comunicação em redes de sensores sem fio: \textit{broadcast}, \textit{convergecast}, \textit{local gossip}. O padrão \textit{broadcast} é geralmente utilizados por estações base (\textit{sink nodes}) para transmitir informações para todos os nós da rede. Pacotes enviados em \textit{broadcast} podem conter requisições por informações dos sensores, atualizações do programa para os nós sensores, pacotes de controle para todo o sistema, etc. \textit{Local gossip} ocorre quando um nó sensor detecta um evento e o comunica a cada um de seus vizinhos localmente (dentro do alcance de seu radio transmissor). Após um grupo de nós detectar um evento, é necessário comunicá-lo ao centro de informações, esse padrão de comunicação é chamado \textit{convergecast} (quando um grupo de sensores se comunica com um sensor específico).

%---------------------------------------------------------------------
\subsection{Classificação de Protocolos de Controle de Acesso ao Meio}
%---------------------------------------------------------------------

A cada dia surgem novas propostas de protocolos MAC que utilizam diferentes técnicas para alcançar melhores resultados em termos de consumo de energia, latência, qualidade de serviço, etc., de acordo com as necessidades de cada aplicação. \citeauthoronline{Chandra_2wireless} (\citeyear{Chandra_2wireless}) classifica os protocolos MAC nas seguintes classes: protocolos de atribuição fixa, protocolos de atribuição por demanda e protocolos de acesso aleatório.

 \subsubsection{Protocolos de atribuição fixa} 
 
 Cada nó recebe uma porção do recursos por um período de tempo considerado longo (da ordem de minutos, horas, ou mesmo maior) e pode utilizá-los de maneira exclusiva naquele período de tempo, ou seja, somente o nó ao qual um recurso foi atribuído pode acessar esse recurso no espaço de tempo que foi para ele reservado. 
 
 A atribuição desses recursos não é feita de maneira permanente devido à necessidade de adaptação a possíveis alterações na topologia da rede (quando ocorre a morte, inserção ou movimentação de nós na rede, por exemplo). Esse tipo de atribuição feito por períodos consideravelmente longos pode causar efeitos negativos à característica de escalabilidade da rede. 
 
 Como exemplos desse tipo de protocolo tem-se o \textit{Time Divided Multiple Access} (TDMA) onde o tempo de uso dos recursos é dividido em \textit{slots} que são atribuídos aos nós da rede \cite{Kulkarni2004}, \textit{Frequency Division Multiple Access} (FDMA) onde a banda de frequência e dividida em vários sub-canais (faixas de frequência) que são atribuídos aos nós da rede \cite{Arms_frequencyagile}.
	
 \subsubsection{Protocolos de atribuição por demanda} 
 
 Nesse tipo de protocolo, ao contrário dos protocolos de atribuição fixa, os recursos são distribuídos entre os nós por um período de tempo relativamente curto (da ordem de dezenas de milissegundos), geralmente o tempo necessário para o envio de um conjunto de pacotes de dados. Caso um nó necessite utilizar algum recurso, este deve fazer uma solicitação a um nó central que seja responsável por coordenar o uso de tal recurso, que por sua vez retorna a confirmação ou não da alocação do recurso e, em caso positivo, uma descrição do recurso alocado. Após essa negociação, o nó solicitante poderá utilizar o recurso de maneira exclusiva. A comunicação para requisição de um recurso é, em geral, feita através de protocolos de acesso aleatório em canais dedicados. Uma outra possibilidade, é deixar a estação central organizar os nós associados a ela. 
 
 Uma restrição no uso desse tipo de organização é que a estação central precisa estar ligada todo o tempo e, portanto, não pode ter restrições de energia. Esse tipo de protocolos pode ser ainda dividido em duas outras subcategorias: protocolos centralizados (onde exista apenas uma estação central) e protocolos distribuídos (onde exista várias estações centrais, cada uma responsável pela alocação dos recursos em uma determinada região da rede). 
 
 Como exemplo temos o \textit{Predictive Demand Assignment Multiple Access} (PRDAMA), que aloca recursos de largura de banda através de estimativa da variação positiva do tráfego da internet para predeterminar a necessidade de recursos das estações \cite{Jiang_apredictive}.

 \subsubsection{Protocolos de acesso aleatório}	

 Em protocolos desse tipo não há coordenação para o uso de recursos. Cada nó que necessite de um recurso (como o meio de comunicação, por exemplo), deve verificar se o recurso está disponível ou tentar acessá-lo diretamente podendo considerar algum parâmetro aleatório, por exemplo, após aguardar um tempo aleatório ou relacionado com o tempo de chegada de pacotes randômicos. Caso o recurso esteja indisponível ou inacessível repete-se o procedimento de acordo com novos valores aleatórios. 
 
 Um exemplo desse tipo de protocolo é o ALOHA, que permite que os nós enviem pacotes no momento em que existam pacotes a serem enviados. Em seguida eles devem aguardar a confirmação do recebimento dos pacotes e, em caso positivo enviar o pacote seguinte (caso haja) ou, no caso contrário, reenviar o pacote \cite{Baccelli06analoha}.
 