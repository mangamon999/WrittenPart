\section{Motivação}

Em redes de sensores, o maior consumo de energia se dá na recepção e sobretudo na transmissão dos dados. Sendo assim é de suma importância garantir que a comunicação entre os nós sensores seja eficiente, evitando problemas que possam resultar na necessidade de retransmissão de pacotes(como no caso da colisão de pacotes por exemplo), bem como a recepção desnecessária de pacotes pelos nós da rede(\emph{overhearing}) ou mesmo a permanência dos receptores em atividade quando não há pacotes sendo enviados ao nó(\emph{idle listening}). Os responsáveis por evitar ou minimizar a ocorrência desses tipos de problemas são os protocolos de controle de acesso ao meio(MAC-\emph{Medium Access Control Protocols}). Esses protocolos determinam, dentre outras coisas, o momento em que cada nó pode começar ou não uma transmissão e os períodos em que o nó deve permanecer desligado ou ligado(ciclo de sono).