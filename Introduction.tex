\section{Introdução}

Redes de sensores sem fio (RSSF) tem sido cada vez mais utilizadas em variados tipos de aplicações em diferentes tipos de ambientes, como no monitoramento de florestas para controle de incêndios, controle de irrigação em grandes plantações, controle de ventilação e condicionamento de ar em grandes edifícios, vigilância em áreas de interesse militar, automação industrial, etc. A cada dia surgem novas tecnologias em sensores, rádio transmissores e outros equipamentos utilizados na construção dos nós sensores que compõem esse tipo de rede, o que tem, cada vez mais, reduzido o custo para produção e, consequentemente, para a aquisição desses equipamentos.

Os nós sensores utilizados para a construção dessas redes são, em geral, alimentados por baterias, o que implica em sérias restrições ao seu tempo de vida útil \cite{Akyildiz2002a}. Considerando o fato de que a maioria das aplicações para esse tipo de rede é feita para utilização em ambientes que dificultam ou impossibilitam a substituição dessas baterias, torna-se necessário ou mais vantajoso substituir os próprios nós por novos. Embora o custo da substituição seja cada vez mais reduzido pelos avanços em tecnologia ainda é mais interessante fazer com que o tempo de vida da rede seja estendido através de mecanismos que possam reduzir o consumo de energia para se evitar uma alta frequência na necessidade de substituição dos nós sensores. Isso é possível através da utilização de protocolos eficientes no consumo de energia. 

\subsection{Motivação}

Em redes de sensores, o maior consumo de energia se dá na recepção e sobretudo na transmissão dos dados. Sendo assim é de suma importância garantir que a comunicação entre os nós sensores seja eficiente, evitando problemas que possam resultar na necessidade de retransmissão de pacotes(como no caso da colisão de pacotes por exemplo), bem como a recepção desnecessária de pacotes pelos nós da rede(\emph{overhearing}) ou mesmo a permanência dos receptores em atividade quando não há pacotes sendo enviados ao nó(\emph{idle listening}). Os responsáveis por evitar ou minimizar a ocorrência desses tipos de problemas são os protocolos de controle de acesso ao meio(MAC-\emph{Medium Access Control Protocols}). Esses protocolos determinam, dentre outras coisas, o momento em que cada nó pode começar ou não uma transmissão e os períodos em que o nó deve permanecer desligado ou ligado(ciclo de sono).

\subsection{Objetivos}

Esse trabalho tem por objetivo investigar mecanismos que promovam redução do consumo de energia em redes de sensores e aumentem seu tempo de vida útil para, ao final, propor um algoritmo que seja capaz de amenizar problemas causados por esses mecanismos, como o grande aumento no tempo de transmissão de pacotes entre pontos distantes na rede. Nas sessões seguintes serão apresentados a estrutura básica de uma rede de sensores, uma visão geral sobre as camadas que compõem a arquitetura dos nós. O foco da investigação são os protocolos da camada de controle de acesso ao meio (em inglês \emph{Medium Access Control - MAC}). Serão apresentados exemplos de protocolos existentes para essa camada incluindo uma classificação e exemplos de cada um deles. Também será apresentado um conceito de auto-organização e sua utilização nesses protocolos. O algoritmo proposto nesse trabalho deverá utilizar técnicas de auto-organização para promover uma organização na rede que possibilite o aumento da velocidade de comunicação entre nós selecionados ao passo que aumenta os períodos de inatividades dos demais nós. Espera-se com isso promover redução do tempo de transmissão de pacotes entre pontos distantes ao fazer com que esses passem necessariamente pelos nós selecionados no passo anterior. 