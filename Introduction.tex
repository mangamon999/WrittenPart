\section{Introdução}

Redes de sensores sem fio(RSSF) tem sido cada vez mais utilizadas em variados tipos de aplicações em diferentes tipos de ambientes, como no monitoramento de florestas para controle de incêndios, controle de irrigação em grandes plantações, controle de ventilação e condicionamento de ar em grandes edifícios, vigilância em áreas de interesse militar, automação industrial, etc. A cada dia surgem novas tecnologias em sensores, radiotransmissores e outros equipamentos utilizados na construção dos nós sensores que compõem esse tipo de rede, o que tem, cada vez mais, reduzido o custo para produção e, consequentemente, para a aquisição desses equipamentos.

Os nós sensores utilizados para a construção dessas redes são, em geral, alimentados por baterias, o que implica em sérias restrições ao seu tempo de vida útil. Considerando o fato de que a maioria das aplicações para esse tipo de rede é feita para utilização em ambientes que dificultam ou impossibilitam a substituição dessas baterias, torna-se necessário ou mais vantajoso substituir os próprios nós por novos. Embora o custo da substituição seja cada vez mais reduzido pelos avanços em tecnologia ainda é mais interessante fazer com que o tempo de vida da rede seja estendido através de mecanismos que possam reduzir o consumo de energia para se evitar uma alta frequência na necessidade de substituição dos nós sensores. Isso é possível através da utilização de protocolos eficientes no consumo de energia. 


