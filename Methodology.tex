\section{Metodologia}

Este trabalho pode ser classificado como pesquisa aplicada por tratar com problemas reais. Espera-se como resultado um protocolo que reduza o tempo de transmissão de informação em redes de sensores sem fio sem redução significativa da vida útil da rede. 

É também pesquisa exploratória, pois parte da solução proposta advém da utilização de algoritmos existente para solução de alguns dos problemas (auto organização na formação de estruturas dentro da rede). Possui também características que permitem defini-la como experimental, pois serão utilizadas simulações para verificação e avaliação do protocolo proposto.

Além disso, se caracteriza ainda como pesquisa quantitativa pela utilização de testes, simulações cujos dados quantitativos resultantes serão avaliados com o objetivo de determinar a eficiência do protocolo proposto nesse trabalho.
